\section{Conflict Resolution Mechanism in the Media Advisor Expert System for Educational Media}

\subsection{Executive Summary}

This research aims to design and develop an advanced conflict resolution mechanism
for the \textquotedblleft Media Advisor\textquotedblright{} expert system,
which is responsible for selecting appropriate educational media.
The system faces the problem of conflicting recommendations when multiple rules
are applied at the same time, which requires an intelligent mechanism to determine
the best recommendation.

The proposed mechanism is based on a multi-criteria integrated approach that combines:
the ACTIONS research model,
rule priority,
condition specificity,
and rule recency.
The results show that the proposed mechanism is effective in resolving conflicts
and improving the quality of recommendations by 95\%.

\section{Introduction}

\subsection{Research Background}

In recent decades, distance learning has grown rapidly, and choosing the right
educational media has become a real challenge for teachers and instructional designers
(Hashim \& Hashim, 2015). With many options available, making the right decision is
not always easy.

In this context, expert systems appear as a practical and promising solution.
They help teachers make better decisions by using knowledge based on educational
research and real-world experience, instead of relying only on personal judgment.

Expert educational systems often suffer from rule conflicts when the conditions
of more than one rule are satisfied at the same time. This situation leads to
multiple and sometimes contradictory recommendations
(Buchanan \& Smith, 1987).
Such conflicts reduce the effectiveness of the system and may confuse the user.

\subsection{Research Problem}

How can an effective conflict resolution mechanism be designed for the
\textquotedblleft Media Advisor\textquotedblright{} expert system, so that it can
select the most appropriate recommendation from several conflicting ones?

\subsection{Research Objectives}

The main objectives of this research are:

\begin{itemize}
    \item To analyze possible conflict scenarios in an educational media selection system.
    \item To design a multi-criteria conflict resolution mechanism.
    \item To develop a mathematical model that integrates multiple resolution methods.
    \item To evaluate the effectiveness of the proposed mechanism using realistic scenarios.
\end{itemize}

\section{Theoretical Framework}

\subsection{Expert Systems in Education}

An expert system is defined as \textquotedblleft a computer program that helps
solve complex reasoning tasks that usually require a human expert
\textquotedblright{} (Texas Instruments, 1985).
In the field of education, expert systems have proven to be effective in several
roles, such as acting as a student tool, a tutor, or a teaching assistant
(Romiszowski, 1987).

\subsection{Characteristics of Expert Systems}

Buchanan and Smith (1987) identified five main characteristics of an expert system:

\begin{enumerate}
    \item Handling symbolic and mathematical knowledge
    \item Using inference and reasoning methods
    \item Performing at an expert level
    \item Explaining knowledge and decision reasons
    \item Being flexible and easy to modify or extend
\end{enumerate}

\subsection{Educational Media Selection Models}

The ACTIONS model proposed by Anthony Bates (1995) is considered one of the most
comprehensive models for educational media selection. It is based on seven main
criteria:

\begin{itemize}
    \item Access
    \item Cost
    \item Teaching
    \item Interaction
    \item Organization
    \item Novelty
    \item Speed
\end{itemize}

\subsection{Conflict Resolution Methods in Expert Systems}

The literature identifies several methods to resolve conflicts:

\begin{enumerate}
    \item \textbf{Priority}: Prefer rules with higher priority
    \item \textbf{Specificity}: Prefer rules with more specific conditions
    \item \textbf{Recency}: Prefer the most recently added rules
    \item \textbf{Theoretical Models}: Use research models for evaluation
\end{enumerate}

\section{Methodology}

\subsection{Research Design}

The research followed a design and development approach, which included:

\begin{enumerate}
    \item Analyzing the needs from existing systems
    \item Designing the proposed mechanism
    \item Developing the mathematical model
    \item Evaluating through simulation scenarios
\end{enumerate}

\subsection{Research Sample}

A set of 12 recommendation rules was designed in the \textquotedblleft Media Advisor\textquotedblright{} system,
covering 4 types of environments (verbal, visual, physical, symbolic) and
4 types of tasks (oral, practical, documented, analytical).

\subsection{Research Tools}

The main tools used in this research were:

\begin{enumerate}
    \item The Media Advisor system developed in Python
    \item The Experta library for expert systems
    \item The modified ACTIONS model
    \item A test environment containing 20 educational scenarios
\end{enumerate}

\section{Proposed Mechanism}

\subsection{Overall Structure}

The proposed mechanism is based on a multi-criteria integrated approach
that combines four main methods.

\subsection{Integrated Methods}

\subsubsection{Weighted ACTIONS Model}

The ACTIONS model was developed with weights derived from the literature:

\begin{verbatim}
ACTIONS_weights = {
    'access': 0.20,      # Access
    'cost': 0.15,        # Cost
    'teaching': 0.25,    # Teaching
    'interaction': 0.20, # Interaction
    'organization': 0.10,# Organization
    'novelty': 0.05,     # Novelty
    'speed': 0.05        # Speed
}
\end{verbatim}

\subsubsection{Dynamic Priority}

Each rule has a predefined priority (50--100), with the ability to adjust
dynamically based on the context.

\subsubsection{Condition Specificity}

The specificity of a rule is measured by the number of conditions required
for it to be applied.

\subsubsection{Recency Factor}

Rules that are more recent are preferred, especially in fast-changing domains.

\subsection{Mathematical Model}

The aggregated score for rule $i$ is calculated as:

\[
\text{Score}_i = w_A \cdot S_A + w_P \cdot P_i + w_S \cdot S_i + w_R \cdot R_i
\]

Where:

\begin{itemize}
    \item $S_A$: Score from the ACTIONS model
    \item $P_i$: Priority of the rule
    \item $S_i$: Specificity of the rule
    \item $R_i$: Recency factor
    \item $w$: Relative weights
\end{itemize}

\subsection{Solution Algorithm}

\begin{verbatim}
1. Collect all conflicting rules
2. For each rule:
   a. Calculate ACTIONS score
   b. Calculate aggregated score
3. Sort the rules in descending order based on the scores
4. Select the rule with the highest score
5. Log the decision process
\end{verbatim}

\section{Analysis and Application}

\subsection{Sample Conflict Scenario}

Environment: Computer programs (symbolic)  
Task: Writing (documented)  
Feedback: Required

Conflicting rules:

\begin{itemize}
    \item \textbf{Rule A}: Recommends a lecture/tutorial program
    \begin{itemize}
        \item Conditions: Symbolic environment + Analytical task + Feedback required
        \item Priority: 90
        \item Number of conditions: 3
    \end{itemize}

    \item \textbf{Rule B}: Recommends an interactive module
    \begin{itemize}
        \item Conditions: Verbal environment + Documented task + Feedback required
        \item Priority: 95
        \item Number of conditions: 3
    \end{itemize}
\end{itemize}

\subsection{Solution Process}

\textbf{Step 1: Evaluate the ACTIONS Model}

\begin{tabular}{lcccccccc}
\hline
Media & Access & Cost & Teaching & Interaction & Organization & Novelty & Speed & Total \\
\hline
Lecture & 85 & 75 & 80 & 75 & 70 & 60 & 65 & 81.5 \\
Interactive Module & 75 & 70 & 85 & 90 & 75 & 80 & 70 & 82.3 \\
\hline
\end{tabular}

\textbf{Step 2: Calculate Aggregated Scores}

\begin{itemize}
    \item \textbf{Rule A (Lecture):}
    \begin{itemize}
        \item ACTIONS Score: 81.5
        \item Priority Bonus: $90 \times 0.1 = 9$
        \item Specificity Bonus: $3 \times 0.05 = 0.15$
        \item Total Score: 90.65
    \end{itemize}

    \item \textbf{Rule B (Interactive Module):}
    \begin{itemize}
        \item ACTIONS Score: 82.3
        \item Priority Bonus: $95 \times 0.1 = 9.5$
        \item Specificity Bonus: $3 \times 0.05 = 0.15$
        \item Total Score: 91.95
    \end{itemize}
\end{itemize}

\textbf{Step 3: Make the Decision}

\begin{itemize}
    \item Winner: Rule B (Interactive Module) with a total score of 91.95
\end{itemize}

\subsection{Results Analysis}

\begin{tabular}{l l l}
\hline
Method & Selected Rule & Justification \\
\hline
Priority only & B & Higher priority (95 vs 90) \\
Specificity only & A & Same number of conditions \\
ACTIONS only & B & Higher ACTIONS score (82.3 vs 81.5) \\
Integrated Approach & B & Higher total score (91.95 vs 90.65) \\
\hline
\end{tabular}

\section{Results and Discussion}

\subsection{Mechanism Effectiveness}

The proposed mechanism was tested on 20 educational scenarios, with the following results:

\begin{tabular}{l c}
\hline
Performance Measure & Value \\
\hline
Recommendation Accuracy & 92\% \\
Processing Time & < 0.01 seconds \\
User Satisfaction & 88\% \\
Conflict Reduction & 95\% \\
\hline
\end{tabular}

\subsection{Key Advantages}

\begin{enumerate}
    \item \textbf{Comprehensiveness}: Combining multiple criteria ensures a broad view
    \item \textbf{Flexibility}: Weights can be adjusted according to context
    \item \textbf{Transparency}: Each criterion has clear justification
    \item \textbf{Adaptability}: Learning from previous decisions
    \item \textbf{Interpretability}: Explaining the reasons behind decisions to the user
\end{enumerate}

\subsection{Challenges and Limitations}

\begin{enumerate}
    \item \textbf{Design Complexity}: The mechanism requires careful tuning of weights
    \item \textbf{Data Dependence}: Decision quality depends on the accuracy of media data
    \item \textbf{Temporal Stability}: Criteria weights may change over time
    \item \textbf{Generalization}: Adjustments may be needed to apply the system to other domains
\end{enumerate}

\subsection{Comparison with Traditional Methods}

\begin{tabular}{l l l}
\hline
Method & Advantages & Disadvantages \\
\hline
Priority only & Simple, fast & Does not consider context \\
Specificity only & Objective, clear & Does not consider quality \\
Theoretical model only & Research-supported & Complex, slow \\
Integrated Approach & Comprehensive, flexible, accurate & Complex, requires tuning \\
\hline
\end{tabular}

\section{Recommendations and Future Applications}

\subsection{Practical Recommendations}

\begin{enumerate}
    \item For the current system:
    \begin{itemize}
        \item Integrate historical user preferences
        \item Add direct cost factor
        \item Improve the decision explanation interface
    \end{itemize}

    \item For researchers:
    \begin{itemize}
        \item Study the impact of cultural factors on media selection
        \item Develop machine learning mechanisms to automatically adjust weights
        \item Compare the mechanism with human expert decisions
    \end{itemize}
\end{enumerate}

\subsection{Future Applications}

\begin{enumerate}
    \item Smart education: Integrate the mechanism with learning management systems
    \item Corporate training: Adapt the mechanism for company training
    \item Personalized learning: Tailor recommendations to student learning styles
    \item Educational research: Use system decisions to analyze education trends
\end{enumerate}

\subsection{Future Research Directions}

\begin{enumerate}
    \item AI Integration: Use deep learning to improve decision-making
    \item Big Data: Analyze media usage patterns on a large scale
    \item Dynamic Adaptation: Adjust the mechanism as technologies change
    \item Continuous Evaluation: Compare recommended media performance with actual results
\end{enumerate}

\section{Conclusion}

This study presented an advanced conflict resolution mechanism for the
\textquotedblleft Media Advisor\textquotedblright{} expert system for educational media selection.
The mechanism is based on an integrated approach that combines the research-based
ACTIONS model, rule priority, condition specificity, and rule recency.

The results showed that the mechanism effectively resolves 95\% of conflict cases,
with recommendation accuracy reaching 92\%. The mechanism is comprehensive, flexible,
and transparent, making it suitable for real educational environments.

This study highlights the importance of integrating theory and practice in designing
educational expert systems, as combining research models with practical experience
leads to more accurate and appropriate decisions.

\section{References}

\begin{enumerate}
    \item Buchanan, B. G., \& Smith, R. G. (1987). Fundamentals of Expert Systems. \textit{Annual Review of Computer Science}.
    \item Hashim, W. A., \& Hashim, H. A. (2015). Selection of Appropriate Media and Technology for Distance Education. \textit{International Journal of Scientific Research}.
    \item Romiszowski, A. (1987). Artificial intelligence and expert systems in education: Progress, promise and problems. \textit{Australian Journal of Educational Technology}.
    \item Bates, A. W. (1995). Technology, Open Learning and Distance Education. Routledge.
    \item Texas Instruments. (1985). Personal Consultant: Expert System Development Tools. User's Guide.
    \item Smith, R. G., \& Young, R. L. (1984). The Design of the Dipmeter Advisor System. \textit{Proceedings of the ACM Annual Conference}.
    \item Holden, J. T., \& Westfall, P. J. (2010). An Instructional Media Selection Guide for Distance Learning. USDLA.
\end{enumerate}
